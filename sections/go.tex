%!TEX root = slides.tex

\section{Go -- Getting Started}

\begin{frame}{Go features}
  \begin{description}
    \item[Concurrency] is builtin with light-weight goroutines, channels.  
    \item[Fast compiling] is a goal.
    \item[Packages] are easily managed and dependencies are quickly resolved.
    \item[Type inference] is available (sometimes).
    \item[C-like] in syntax. 
    \item[Tools] like go fmt and godoc are builtin. 
    \item[Garbage collection] is builtin.
  \end{description}
\end{frame}

\begin{frame}[fragile]{Hello, World!}
  \begin{minted}{go}
package main

import "fmt"

func main() {
  fmt.Println("Hello, world!")
}
  \end{minted}
  \citeurl{gobyexample.com/hello-world}
\end{frame}

\begin{frame}{go build}
  \begin{description}
    \item[go build] is the Go compiler.
    \item[go build hello.go] created an executable called hello (or hello.exe on windows).
    \item[Dependencies] are automatically built.
    \item[go run] is an alternative that also runs the program after.
    \item[Building] is fast in Go.
  \end{description}
  
  \citeurl{golang.org/cmd/go/}
\end{frame}


\begin{frame}[fragile]{Functions}
  \begin{minted}{go}
package main

import "fmt"

func add(x int, y int) int {
  return x + y
}

func main() {
  fmt.Println(add(42, 13))
}
  \end{minted}
  \citeurl{tour.golang.org/basics/4}
\end{frame}


\begin{frame}[fragile]{for loops}
  \begin{minted}{go}
package main

import "fmt"

func main() {
  sum := 0
  for i := 0; i < 10; i++ {
    sum += i
  }
  fmt.Println(sum)
}
  \end{minted}
  \citeurl{tour.golang.org/flowcontrol/1}
\end{frame}


\begin{frame}[fragile]{if and else}
  \begin{minted}{go}
func pow(x, n, lim float64) float64 {
  if v := math.Pow(x, n); v < lim {
    return v
  } else {
    fmt.Printf("%g >= %g\n", v, lim)
  }
  // can't use v here, though
  return lim
}
  \end{minted}
  \citeurl{tour.golang.org/flowcontrol/7}
\end{frame}

\begin{frame}[fragile]{Goroutines}
  \begin{minted}{go}
func say(s string) {
  for i := 0; i < 5; i++ {
    time.Sleep(100 * time.Millisecond)
    fmt.Println(s)
  }
}

func main() {
  go say("world")
  say("hello")
}
  \end{minted}
  \citeurl{tour.golang.org/concurrency/1}
\end{frame}